Time series models of hydraulic heads represent salient information for decision-makers. Time series models using predefined response functions are a fast, data-driven method to simulate groundwater head dynamics caused by various stresses such as areal recharge, pumping, or surface water fluctuations. These stresses contain errors originating in the observation process, which affect the simulated head dynamics and their uncertainty. Other factors that influence the performance of time series models are errors in parameter estimation, simplifications in the model concept, and numerical errors in the solution of the time series model. All these factors obviously affect the uncertainty of the time series model. The main objective of this study is to assess the impact of errors in the stresses on the residuals and uncertainty of time series models. The effect of these observation errors is assessed using synthetic head series generated with a physics-based, variably saturated numerical groundwater model. The groundwater model is a controlled environment where all fluxes and head values, i.e., the "truth'', are known, different aquifer properties can be used, and any complexity or processes deemed important can be added. First, the effect of simplifications in the model concept (model error) may be quantified by simulating the synthetic head series with time series models using the same stresses that were used to create the synthetic head series. Next, the impact of errors in the stresses on the simulation of the head and the estimated uncertainty can be assessed. This study presents scenarios where errors are introduced to the observations, e.g., by adding different types of noise to the stresses or by using observations of stresses from another geographical location. The results provide valuable insights into improving the reliability and robustness of time series models, particularly in scenarios where uncertainty estimates are critical to decision-making processes.